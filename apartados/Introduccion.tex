\chapter{Introducción}

Este Trabajo de Fin de Grado ha sido realizado íntegramente por mi persona, Adrián Morente Gabaldón, en colaboración con la empresa \textit{Synergy Movement Technologies 3000 S.L.} durante el período laboral mantenido con dicha compañía. El proyecto nace como solución propuesta a una parte del desarrollo de DYNAsystem \cite{dynasystem-web}, un dinamómetro funcional en forma de estación orientado al rendimiento deportivo y al tratamiento fisioterapéutico, el cual concentra la mayoría de los modos de entrenamiento que pueden encontrarse en un gimnasio, embebidos en forma de \textit{``caja negra''} y con un factor de forma y tamaño fácilmente integrables en cualquier entorno. Además, dada la naturaleza electrónica de estos dispositivos, tienen la capacidad de mostrar en tiempo real los datos que están siendo producidos por el usuario (como las fuerzas y velocidades medias y máximas, trabajo, número de repeticiones, tiempo de entrenamiento, etc.). Por otro lado, el control de dicho dispositivo se realiza a través de una aplicación embebida con interacción táctil, desde la que se configuran todos los ejercicios y rutinas deseadas por el usuario.\\

Dicha empresa llegó a vender a nivel internacional; pero atravesó diferentes fases temporales, cambiando de administración y nombre en un par de ocasiones, al igual que el producto (que también pasó por diferentes implementaciones), lo que derivó en la decisión de paralizar la venta de estos dispositivos. De esta forma sería factible realizar una re-formulación al completo de sus máquinas, mejorando tanto en \textit{potencia}, \textit{funcionalidad} y \textit{estética} como en \textit{reducción de costes y consumo}.\\

Una de las medidas que podrían ayudar a este rediseño sería la implantación de una infraestructura virtual diferente: un sistema operativo adaptado por y para la compañía, moderno y libremente personalizable. De esta forma, además de ejecutar la aplicación embebida que necesita el sistema para ser utilizado, el dispositivo podría constar de valores añadidos como animaciones de carga o \textbf{actualizaciones remotas} para solventar problemas de software y añadir nuevas funcionalidades a lo largo del tiempo.\\

En este documento hablaremos de los requisitos que se plantearon inicialmente para dicha infraestructura del producto (obviando otros aspectos del producto como el diseño estructural o la implementación de la propia aplicación embebida), trataremos el análisis que realicé y discutiremos las decisiones tomadas por mi parte para solucionar los problemas a lo largo del desarrollo. Finalmente, estimaremos un presupuesto del proyecto para éste y cualquier otro propósito general.\\

Cabe destacar que el proyecto es libre \textit{(bajo la licencia LGPL v3.0)} y está alojado en mi perfil de \textit{Github} \cite{github-adrian} con el consentimiento de la empresa, lo cual hay que agradecer por todas las ventajas que ello comporta.

\newpage