\chapter*{}

\cleardoublepage
\thispagestyle{empty}

\begin{center}
	{\large\bfseries Meta-Dynasystem}\\
\end{center}
\begin{center}
	Adrián Morente Gabaldón\\
\end{center}

%\vspace{0.7cm}
\noindent{\textbf{Palabras clave}: GNU/Linux, distribución Linux, dispositivo embebido, actualizaciones OTA, Yocto Project, Mender.io, Dynasystem}\\

\vspace{0.7cm}
\noindent{\textbf{Resumen}}\\

Actualmente la sociedad se encuentra cada vez más sumida en un ecosistema total de computación ubicua, rodeándose de pequeños computadores con funciones muy específicas y determinadas, cuyas interacciones con el entorno hacen que su naturaleza tecnológica pase desapercibida para los humanos. Se tratan de los llamados \textbf{sistemas embebidos}, que plantean nuevos retos al gremio de ingenieros de cualquier tipo, quienes intentan maximizar la productividad e innovar en funcionalidad además de ahorrar consumo y precios de coste.\\

El proyecto \textit{\textbf{Meta-DYNAsystem}} nace como solución propuesta a la implementación de una \textbf{infraestructura virtual} para un sistema dedicado de este tipo.\\

Persigue el desarrollo de una nueva distribución propia basada en GNU/Linux que dote al dispositivo físico de ligereza, eficacia y el rendimiento necesario para embeber una aplicación de \textbf{tiempo real}. Además, presta especial atención a detalles estéticos que hacen más natural esta interacción con el sistema y presenta la información de forma más atractiva para el usuario.\\

Finalmente, intenta innovar en el uso de distribuciones embebidas dotándose de \textbf{actualizaciones \textit{Over-the-Air}} centralizadas a través de un servidor dedicado.\\

\cleardoublepage
\thispagestyle{empty}

\begin{center}
	{\large\bfseries Meta-DYNAsystem}\\
\end{center}
\begin{center}
	Adrián Morente Gabaldón\\
\end{center}

%\vspace{0.7cm}
\noindent{\textbf{Keywords}: GNU/Linux, Linux distro, embedded device, OTA updates, Yocto Project, Mender.io, Dynasystem}\\

\vspace{0.7cm}
\noindent{\textbf{Abstract}}\\

Nowadays, society is increasingly immersed in an ubiquitous computing ecosystem, surrounding itself with small computers with very specific and determined functions, whose interactions with the whole environment make their technological origin goes unnoticed by humans. These devices are known as \textbf{embedded systems}, which pose new challenges to the engineers' guild of any kind, who try to maximize productivity and innovate in functionality, in addition to saving consumption and cost prices.\\

The \textbf{\textit{Meta-DYNAsystem}} project was born as a proposed solution to the implementation of a \textbf{virtual infrastructure} for a dedicated system of this kind.\\

It pursues the development of a new proprietary GNU/Linux-based distribution that endows the physical device with lightness, effectiveness and the necessary performance to embed a real-time application. Moreover, it attends to aesthetical details which make this interaction with the system more natural and it shows the information in a more attractive way to the user.\\

Finally, it tries to innovate in the use of embedded distributions, equipping itself with \textbf{\textit{Over-the-Air} updates} centralized through a dedicated server.

\chapter*{}
\thispagestyle{empty}

\noindent\rule[-1ex]{\textwidth}{2pt}\\[4.5ex]

Yo, \textbf{Adrián Morente Gabaldón}, alumno de la titulación Grado en Ingeniería Informática de la \textbf{Escuela Técnica Superior
de Ingenierías Informática y de Telecomunicación de la Universidad de Granada}, con DNI 77139229N, autorizo la
ubicación de la siguiente copia de mi Trabajo Fin de Grado en la biblioteca del centro para que pueda ser
consultada por las personas que lo deseen.

\vspace{6cm}

\noindent Fdo: Adrián Morente Gabaldón

\vspace{2cm}

\begin{flushright}
Granada a 18 de junio de 2018.
\end{flushright}


\chapter*{}
\thispagestyle{empty}

\noindent\rule[-1ex]{\textwidth}{2pt}\\[4.5ex]

D. \textbf{Juan Julián Merelo Guervós}, Profesor del Área de Ingeniería Informática del Departamento de Arquitectura y Tecnología de Computadores de la Universidad de Granada.

\vspace{0.5cm}

\textbf{Informa:}

\vspace{0.5cm}

Que el presente trabajo, titulado \textit{\textbf{Meta-Dynasystem}}, ha sido realizado bajo su supervisión por \textbf{Adrián Morente Gabaldón}, y autorizo la defensa de dicho trabajo ante el tribunal que corresponda.

\vspace{0.5cm}

Y para que conste, expide y firma el presente informe en Granada a 18 de junio de 2018.

\vspace{1cm}

\textbf{El director:}

\vspace{5cm}

\noindent \textbf{Juan Julián Merelo Guervós}

\chapter*{Agradecimientos}

A mi madre, padre y hermana por haber hecho posible que siga estudiando en lugar de dejarlo. Este grado es de los cuatro.\\

A mis amigos, por hacer que salgamos todos adelante mientras bromeamos con no movernos del sitio.\\

A mi tutor por su representación y ayuda indispensable en cualquier momento en que lo necesite, a lo largo de toda la carrera.\\

A todo el equipo de \textit{Symotech} por recibirme en el equipo, escucharme como a uno más y permitirme tomar decisiones autónomas que me maduren como profesional.\\